\documentclass[a4paper,12pt]{article}
\usepackage[T2A]{fontenc} 
\usepackage[utf8]{inputenc}
\usepackage[russian]{babel}
\usepackage{graphicx}
\usepackage{amsmath}
\usepackage{cite}
\usepackage{hyperref}

\title{Литературный обзор по физическому моделированию}
\author{Коваль Константин}
\date{\today}

\begin{document}

\maketitle

\begin{abstract}
    В данном литературном обзоре рассматриваются основные подходы и методы физического моделирования, а также их применение в различных областях моделирования аккустических волн и пламени.
\end{abstract}

\section{Введение}
Данный проект был создан на основе задачи данной мне, основная его задача - это изучения влияние зауковых волн на пламя.
\newline
\textbf{Цель проекта:}\begin{enumerate}
    \item Изучение физических характеристик пламени
    \item Анализ звуковых параметров
    \item Математическое и физическое моделирование
\end{enumerate}
\subsection{Актуальность}
Инновационные технологии: Внедрение новых методов, таких как акустическое гашение пламени, может стать основой для разработки инновационных технологий в различных отраслях, включая энергетику, транспорт и строительство. Это может привести к созданию более безопасных и эффективных систем управления горением
\newpage
\section{Обзор литературы}
\subsection{Основные направления}
%Здесь описываются ключевые направления физического моделирования, такие как численные методы, аналитические подходы и экспериментальные данные.
\begin{enumerate}
    \item \textbf{Численные методы}\begin{itemize}
        \item \textbf{Численные методы}, такие как: \begin{itemize}%методы используются для решения дифференциальных уравнений
            \item Интегрирование методами\begin{itemize}
                \item Метод Эйлера
                \item Модиффицированный метод Эйлера
            \end{itemize}
            \item Дифференцирование
        \end{itemize}
        \item \textbf{Методы конечных разностей и элементов}%Эти подходы используются для решения уравнений в частных производных
        \item \textbf{Оптимизация}\begin{itemize} %используются для нахождения экстремумов функций
            \item Градиентные методы
            \item Эволюционные алгоритмы
        \end{itemize}
    \end{itemize}
    \item \textbf{Аналитические методы}\begin{itemize}
        \item \textbf{Математический анализ}
        \item \textbf{Аппроксимация и интерполяция}
        \item \textbf{Математическое моделирование}\\ Для более наглядной демонстрации создаеться математическая модель, те симуляция с визуализацией, которая демонстрирует физические явления
    \end{itemize}
\end{enumerate}

\section{Описание проекта}
\subsection{Характеристики пламени}
\begin{itemize}
    \item Темпиратура пламени(Влияет на его стабильность)
    \item Состав горючего(Может очень сильно повлиять на восприимчивость к звуковым волнам)
    \item Размер и форма пламени(также может влиять на восприимчивость к звуковым колебаниям.Меньшие пламя могут быть более чувствительными к изменениям)
\end{itemize}
\subsection{Применение}
Применение тушения пламени с помощью звука. Использование этого эффекта можно предложить в местах, где не доступну привычные нам методы тушения(К таким местам можно отнести сервера, и другую технику которую нельзя тушить обычной водой, для сохранения работоспособности)
\section{Результаты иследований}
%Представление результатов, полученных в ходе литературного обзора, включая графики и таблицы.
\begin{itemize}
    \item Влияние физических характеристик пламени: Температура, состав газов и размеры пламени
    \item Звуковые параметры: Частота, интенсивность и длительность звуковых волн
    \item Математическое моделирование: Разработка математических моделей, описывающих взаимодействие звука и пламени
    \item Практическое применение: Результаты исследования могут быть использованы в различных отраслях, включая промышленность и пожарную безопасность.
\end{itemize}
\section{Заключение}
%Краткое резюме основных выводов и предложений для будущих исследований.
В ходе исследования влияния звука на гашение пламени было рассмотрено множество аспектов, включая физические характеристики пламени, параметры звуковых волн и методы математического моделирования. Акустические технологии представляют собой перспективный подход к контролю процессов горения, что имеет важное значение для повышения безопасности.\\
Данное исследование открывает множество направлений для будущих исследований. Возможности включают изучение влияния различных типов звуковых волн, разработку более сложных математических моделей и проведение экспериментов в реальных условиях. Также стоит рассмотреть интеграцию акустических технологий с другими методами гашения пламени для создания более эффективных систем.
\newpage
\section*{Список литературы}
\begin{enumerate}
    \item \href{http://www.ict.nsc.ru/sites/default/files/discouncil/Enlist/Oold_Ref/cherny.pdf}{ЧИСЛЕННЫЕ МЕТОДЫ МОДЕЛИРОВАНИЯ И ОПТИМИЗАЦИИ В ГИДРОДИНАМИКЕ ТУРБОМАШИН}\\
    \item \href{https://dep_vipm.pnzgu.ru/files/dep_vipm.pnzgu.ru/konference/achmm2016.pdf}{Аналитические и численные методы моделирования естественно-научных и социальных проблем}\\
    \item \href{https://staff.tiiame.uz/storage/users/4/books/Ghrf3Zz0xenlYSpVTogElcVOyHxu97JG4NOr1Qwu.pdf}{Основы аналитического анализа}\\
    \item \href{https://kpfu.ru/staff_files/F1845062855/Analiticheskie.i.chislennye.metody.pdf}{АНАЛИТИЧЕСКИЕ И ЧИСЛЕННЫЕ МЕТОДЫ}\\
\end{enumerate}
\bibliographystyle{plain}
\bibliography{literature}
\end{document}