\documentclass[a4paper,11pt]{article}
\usepackage[left=2cm,right=2cm,
top=2cm,bottom=2cm,bindingoffset=0cm]{geometry}

%%% Работа с русским языком
\usepackage{cmap}					% поиск в PDF
\usepackage{mathtext} 				% русские буквы в формулах
\usepackage[T2A]{fontenc}			% кодировка
\usepackage[utf8]{inputenc}			% кодировка исходного текста
\usepackage[english,russian]{babel}	% локализация и переносы

%%% Дополнительная работа с математикой
\usepackage{amsmath,amsfonts,amssymb,amsthm,mathtools} % AMS
\usepackage{icomma} % "Умная" запятая: $0,2$ --- число, $0, 2$ --- перечисление

%% Номера формул
%\mathtoolsset{showonlyrefs=true} % Показывать номера только у тех формул, на которые есть \eqref{} в тексте.
%\usepackage{leqno} % Нумерация формул слева

%% Свои команды
%\DeclareMathOperator{\sgn}{\mathop{sgn}}

%% Перенос знаков в формулах (по Львовскому)
%\newcommand*{\hm}[1]{#1\nobreak\discretionary{}
	%	{\hbox{$\mathsurround=0pt #1$}}{}}

%%% Работа с картинками
\usepackage{graphicx}  % Для вставки рисунков
\graphicspath{{images/}{images2/}}  % папки с картинками
\setlength\fboxsep{3pt} % Отступ рамки \fbox{} от рисунка
\setlength\fboxrule{1pt} % Толщина линий рамки \fbox{}
\usepackage{wrapfig} % Обтекание рисунков текстом

%%% Работа с таблицами
\usepackage{array,tabularx,tabulary,booktabs} % Дополнительная работа с таблицами
\usepackage{longtable}  % Длинные таблицы
\usepackage{multirow} % Слияние строк в таблице

%%% Теоремы
%\theoremstyle{plain} % Это стиль по умолчанию, его можно не переопределять.
%\newtheorem{theorem}{Теорема}[section]
%\newtheorem{proposition}[theorem]{Утверждение}

%\theoremstyle{definition} % "Определение"
%\newtheorem{corollary}{Следствие}[theorem]
%\newtheorem{problem}{Задача}[section]

%\theoremstyle{remark} % "Примечание"
%\newtheorem*{nonum}{Решение}

%%% Программирование
%\usepackage{etoolbox} % логические операторы

%%% Страница
%\usepackage{extsizes} % Возможность сделать 14-й шрифт
%\usepackage{geometry} % Простой способ задавать поля
%\geometry{top=25mm}
%\geometry{bottom=35mm}
%\geometry{left=35mm}
%\geometry{right=20mm}
%
%\usepackage{fancyhdr} % Колонтитулы
% 	\pagestyle{fancy}
%\renewcommand{\headrulewidth}{0pt}  % Толщина линейки, отчеркивающей верхний колонтитул
% 	\lfoot{Нижний левый}
% 	\rfoot{Нижний правый}
% 	\rhead{Верхний правый}
% 	\chead{Верхний в центре}
% 	\lhead{Верхний левый}
%	\cfoot{Нижний в центре} % По умолчанию здесь номер страницы

\usepackage{setspace} % Интерлиньяж
%\onehalfspacing % Интерлиньяж 1.5
%\doublespacing % Интерлиньяж 2
%\singlespacing % Интерлиньяж 1

\usepackage{lastpage} % Узнать, сколько всего страниц в документе.

\usepackage{soul} % Модификаторы начертания

\usepackage{hyperref}
\usepackage[usenames,dvipsnames,svgnames,table,rgb]{xcolor}
\hypersetup{				% Гиперссылки
	unicode=true,           % русские буквы в раздела PDF
	pdftitle={Заголовок},   % Заголовок
	pdfauthor={Автор},      % Автор
	pdfsubject={Тема},      % Тема
	pdfcreator={Создатель}, % Создатель
	pdfproducer={Производитель}, % Производитель
	pdfkeywords={keyword1} {key2} {key3}, % Ключевые слова
	colorlinks=true,       	% false: ссылки в рамках; true: цветные ссылки
	linkcolor=red,          % внутренние ссылки
	citecolor=black,        % на библиографию
	filecolor=magenta,      % на файлы
	urlcolor=cyan           % на URL
}

\usepackage{csquotes} % Еще инструменты для ссылок

%\usepackage[style=authoryear,maxcitenames=2,backend=biber,sorting=nty]{biblatex}

\usepackage{multicol} % Несколько колонок

%\usepackage{tikz} % Работа с графикой
%\usepackage{pgfplots}
%\usepackage{pgfplotstable}

\usepackage{titlesec}
\usepackage[most]{tcolorbox} % для управления цветом

\definecolor{block-gray}{gray}{1.00} % уровень прозрачности (1 - максимум)
\newtcolorbox{myquote}{colback=block-gray,grow to right by=-4mm,grow to left by=-4mm,
	boxrule=1pt,boxsep=0pt,breakable} % настройки области с изменённым фоном

\renewcommand{\phi}{\ensuremath{\varphi}}
\renewcommand{\epsilon}{\ensuremath{\varepsilon}}
\renewcommand{\kappa}{\ensuremath{\varkappa}}


\begin{document}

\begin{multicols}{2}
Группа ТЮФ(Ебанаты) 

Студенты Коваль К. 

Преподаватель Мокаренко А. О.

К работе не допущен

Работа выполнена - всегда 

Отчет принят 
\end{multicols}
\renewcommand{\thesection}{\Alph{section}}

\noindent\makebox[\linewidth]{\rule{\textwidth}{2pt}}

\begin{center}
    \textbf{\huge Техническое задание по задаче ТЮФ(2024/25г) №12(Звук против пламени) \normalsize}
    
    \Large\textbf{--------------------------------------------------------------------------------------------------------}
\end{center}
\section{Оглавление}
\begin{enumerate}
	\item Задача 
	
	Небольшое пламя можно погасить с помощью звука.
    Исследуйте параметры пламени и характеристики звука, определяющие, погаснет ли пламя.

	\item Условный план работы 
	\begin{enumerate}
		\item Найти необходимую докумантацию, информацию, формулы и тд.
		\item Создать моделирование пламени и его характеристик 
		\item Создать моделирование звука и его характеристик
		\item Дописать в моделяцию огня, под воздействием звуковых волн 
	\end{enumerate}
	\item Объект исследования:
        Как влияет частота, длины волны, громкость звука на пламя
	\item Метод экспериментального исследования:
        Еще б я знал 
	\item Рабочие формулы и исходные данные
	
	\begin{multicols}{2}

   \begin{equation}\label{eq:i1} 
     V = F \cdot \lambda
   \end{equation}  
    
    \begin{equation}\label{eq:i2}  
        I=\frac{P}{A}
    \end{equation}    
    
    \begin{equation}  \label{eq:i3} 
       L = 10 \log_{10}(\frac{p^2}{p^2_0})
    \end{equation} 
    
    \begin{equation}  \label{eq:i4} 
       A=\frac{F}{k}
    \end{equation} 
    \end{multicols}
        
    \end{enumerate}

\section{Экспериментальная часть}

\renewcommand{\thesubsection}{\arabic{subsection}}

\subsection{Параметры влияющие на установку}
\begin{enumerate}
    \item Параметры пламени:\begin{enumerate}
        \item Частота пульсаций пламени(При совпадение частот пульсации и звковых волн)
        \item Размер пламени
        \end{enumerate}
    \item Параметры звуковых волн:\begin{enumerate}
        \item Интенсивность звука($I = 150\pm 50$ дБ)
        \item Частота звука (P.S. Спизжинные частоты 500-600 Гц)
        \item Направленность потока 
        \item Тип звуковой волны
    \end{enumerate}
\end{enumerate}
\subsection{Результаты прямых измерений}


\subsection{Результаты рассчётов, графики}


\subsection{Выводы, оценка результатов, оценка погрешностей}


\subsection{Замечания преподавателя}

\section{Приложение}
\renewcommand{\thesubsection}{\alph{subsection}}

\noindent

%\subsection{Графики}


\noindent

%\subsection{Таблицы}


\noindent

%Полезные шаблоны
%Случайная погрешность

%\[
%\sigma = \sum\limits_{i = 1}^n (\sqrt{\dfrac{(x_i - \langle x \rangle)^2}{N(N-1)}})
%\]

%Метод наименьших квадратов
%
%Для нахождения коэффициентов A, C зависимости $Y(X) = AX + C$
%\begin{multicols}{2}
%
%\begin{equation}
%	\overline{X} = \dfrac{1}{N} \sum\limits_{i = 1}^N X_i
%\end{equation}
%
%\begin{equation}
%	D = \sum\limits_{i = 1}^N (X_i - \overline{X})^2
%\end{equation}
%
%\begin{equation}
%	A = \dfrac{1}{D} \sum\limits_{i = 1}^N (X_i - \overline{X})Y_i
%\end{equation}
%
%\begin{equation}
%	\Delta A = \sqrt{\dfrac{E}{D}}
%\end{equation}
%
%\begin{equation}
%	\overline{Y} = \dfrac{1}{N} \sum\limits_{i = 1}^N Y_i
%\end{equation}
%
%\begin{equation}
%	E = \dfrac{1}{N-2} \sum\limits_{i = 1}^N(Y_i - AX_i - C)^2	
%\end{equation}
%
%\begin{equation}
%	C = \overline{Y} - A \overline{X}
%\end{equation}
%
%\begin{equation}
%	\Delta C = \sqrt{\left(\dfrac{1}{N} + \dfrac{\overline{X}^2}{D} \right) \cdot E}
%\end{equation}
%
%\end{multicols}

\end{document}